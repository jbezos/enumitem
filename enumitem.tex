%
% Copyright (C) 2003-2019 Javier Bezos http://www.texnia.com
%
% This file may be distributed and/or modified under the conditions of
% the MIT License. A version can be found at the end of this file.
%
% Repository: https://github.com/jbezos/enumitem
%

\documentclass[a4paper]{ltxguide}

\raggedright
\parskip=.8ex
\advance\oddsidemargin-.5cm
\advance\textwidth1cm
\addtolength{\textheight}{2cm}
\addtolength{\topmargin}{-1.5cm}

\usepackage{xcolor,bera}

\definecolor{notes}{rgb}{.75, .3, .3}%

\makeatletter
\newenvironment{desc}
  {\if@nobreak
     \vskip-\lastskip
     \vspace*{-2.5ex}%
   \fi
   \decl}
  {\enddecl}
\def\@begintheorem#1#2{%
  \list{}{}%
  \global\advance\@listdepth\m@ne
  \item[{\sffamily\bfseries\color{notes}\MakeUppercase{#1}}]}%
\makeatother
\newtheorem{warning}{Warning}
\newtheorem{note}{Note}
\newtheorem{example}{Example}
\makeatother

\usepackage{framed}
\definecolor{shadecolor}{rgb}{0.96,0.96,0.93}
\let\bblxv\verbatim
\let\bblexv\endverbatim
\def\verbatim{\begin{shaded*}\bblxv\vskip-\baselineskip\vskip2.5\parsep}
\def\endverbatim{\bblexv\vskip-2\baselineskip\end{shaded*}}

\newcommand\New[1]{%
  \colorbox[rgb]{.87, .9, .83}{New #1}\enspace\ignorespaces}

\usepackage{hyperref}

\title{Customizing lists\\with the\\\textsf{enumitem} package}

\author{Javier Bezos\footnote{For bug reports, comments and
suggestions go to \href{http://www.texnia.com/enumitem.html}%
{\texttt{http://www.texnia.com/enumitem.html}}.}}

\date{Version 3.9\\2025/01/05}

\IfFileExists{enumitem.sty}{\usepackage{enumitem}}{}
\IfFileExists{layouts.sty}{\usepackage{layouts}}{}

\addtolength{\topmargin}{-3pc}
\addtolength{\textwidth}{6pc}
\addtolength{\oddsidemargin}{-2pc}
\addtolength{\textheight}{7pc}

\raggedright
\parindent1.8em
\parskip0pt

\begin{document}

\maketitle
\tableofcontents

\newpage

\vspace*{1cm}

\begin{note}
  Changes and new features with relation to version 3.5 are highlighted
  with \New{X.X}\hspace{-.5em}. The most recent features could be still
  unstable. Please, report any issues you find on
  \texttt{https://github.com/jbezos/enumitem/issues}, which is better
  than just complaining on an e-mail list or a web forum. Forking
  and pull requests are welcome.
\end{note}

\begin{warning}
  Version 3.6 introduced two new keys: |left| and |first|. If your
  documents define some series with these names, an error is raised.
  Just rename them, or set the package option |series=override|.
\end{warning}

\section{Introduction}

This package provides most of the flexibility you may want to customize
the three basic list environments (|enumerate|, |itemize| and
|description|) and to design your own lists, with a |<key>=<value>|
syntax:
\begin{itemize}
\item Fancy labels and fancy refs, including a syntax similar to that
in the \textsf{enumerate} package.

\item Alternative ways for positioning the label, with a new 
  parameter (|labelindent|) and a tab-like setting (|left|).

\item Settings applied globally or only in one of the three types or
even in a single list (including |\topsep|).

\item Inline lists.

\item Several description styles (which fix some bad spacing, too).

\item |trivlist|s properly formatted.

\item Gathering of lists to be treated like a unit, as well as counter 
resuming.

\end{itemize}

In the interface a sort of ``inheritance'' is used. You can set
globally the behavior of lists and then override several parameters of,
say, |enumerate| and then in turn override a few parameters in a
particular instance. The values will be searched in the hierarchy.

The package extends the syntax of the lists to allow an optional
argument where a set of parameters in the form |key=value| are
available. These keys are equivalent to the well known list parameters.
Please, see a \LaTeX{} manual for a description of them. Next sections
explain the extensions provided by \textsf{enumitem}.
\begin{itemize}
\item
Vertical spacing:
\begin{itemize}
\setlength{\itemsep}{0pt}
\item |topsep|
\item |partopsep|
\item |parsep|
\item |itemsep|
\end{itemize}
\item
Horizontal spacing:
\begin{itemize}
\setlength{\itemsep}{0pt}
\item |leftmargin|
\item |rightmargin|
\item |listparindent|
\item |labelwidth|
\item |labelsep|
\item |itemindent|
\end{itemize}
\end{itemize}

\begin{example}
  A straightforward example is:
\begin{verbatim}
\begin{itemize}[itemsep=1ex, leftmargin=1cm]
\end{verbatim}
\end{example}

\begin{note}
  A way to see how these parameters work is with the |layouts| package
  (the manual is named |layman.pdf|).
  \ifx\listdiagram\notundefined\else
    See figure \ref{f.drawlist}.
    \begin{figure}
      \centering
      \listdiagram
      \caption{List parameters}\label{f.drawlist}
    \end{figure}
  \fi
\end{note}

\section{Quick reference}

Some common settings. See the manual below for details.

\begin{itemize}
\def\verbatim{\begin{shaded*}\bblxv}
\def\endverbatim{\bblexv\vskip-1.6\baselineskip\end{shaded*}}

\item To remove the vertical space altogether in a list:
\begin{verbatim}
\begin{enumerate}[nosep]
\end{verbatim}

\item To remove the vertical space altogether in \textit{all} lists:
\begin{verbatim}
\setlist{nosep}
\end{verbatim}

\item To start the label at the margin and the item text at the 
  current parindent:
\begin{verbatim}
\begin{enumerate}[left=0pt .. \parindent]
\end{verbatim}

\item To configure the labels like in \textsf{enumerate}: include the
package option |shortlabels| and then, as a first element, write your
label:
\begin{verbatim}
\begin{enumerate}[(1)]
\end{verbatim}

\item To continue the previous list, after a ``pause'':
\begin{verbatim}
\begin{enumerate}[resume*]
\end{verbatim}

\item To use the three basic lists in line: just add the package option 
|inline| and then the environments |enumerate*|, |itemize*| and 
|description*|.

\item To set a numeric label with parenthesis, but a cross-reference
without them:
\begin{verbatim}
\begin{enumerate}[label=(\arabic*), ref=\arabic*]
\end{verbatim}

\end{itemize}

\section{Keys}

This section describes the keys in displayed lists. Most of them are
available in inline lists, where further keys are available (see
\ref{s.inline}).

\begin{warning}
  If the value is completely enclosed in braces, they are stripped off.
  If you want the braces, they must be duplicated. This is the default
  behavior of \textsf{keyval}, which \textsf{enumitem} just emulates.
\end{warning}

\subsection{Label and cross references format}

\begin{desc}
|label=<commands>|
\end{desc}

Sets the label to be used in the current level. A set of starred
versions of |\alph|, |\Alph|, |\arabic|, |\roman| and |\Roman|, without
argument stand for the current counter in
|enumerate|.\footnote{Actually, the asterisk is currently the argument
but things may change. Consider them as starred variants and follow the
corresponding syntax.} It works with |\value|, too (provided the widest
label is not to be computed or |widest*| is used, see below).

\begin{note}
  If you prefer setting labels like the \textsf{enumerate} package, use
  ``short labels'' (see section \ref{s.short}).
\end{note}

\begin{example}
  The following prints \textit{a}), \textit{b}), and so on (this is a
  standard style in Spanish, and formerly used by Chicago, too).
\begin{verbatim}
\begin{enumerate}[label=\emph{\alph*})]
\end{verbatim}
\end{example}

\begin{warning}
  The value of |label| is a moving argument, and fragile commands must
  be protected \textit{except} the counters.  Because of that, use of
  |\value| is somewhat tricky, because |\the| or |\ifnum|
  expects an actual value, which is not the case when |label| is
  being processed to replace internally the |*| by the form with
  the counter argument.  The best solution is usually encapsulating the
  logic inside a new ``counter'' with the help of
  |\AddEnumerateCounter|.\footnote{Which is admittedly somewhat
  convoluted.  A better way to accomplish this is on the way.}
\end{warning}

\begin{example}
  A fancier example
  (which looks ugly, but it is intended only to illustrate what is
  possible; requires \textsf{color} and \textsf{pifont}):
\begin{verbatim}
\begin{enumerate}[label=\protect\fcolorbox{blue}{yellow}{\protect\ding{\value*}}]
\end{verbatim}
\end{example}

\begin{desc}
|label*=<commands>|
\end{desc}

Like |label| but its value is appended to the parent label. For
example, the following defines a |legal| list (1., 1.1., 1.1.1., and so
on):
\begin{verbatim}
\newlist{legal}{enumerate}{10}
\setlist[legal]{label*=\arabic*.}
\end{verbatim}

\begin{desc}
|ref=<commands>|
\end{desc}

By default, |label| sets also the form of cross references and
|\the...| (overriding the settings in parent hierarchical
levels), but you can define a different format with this key.  For
example, to remove the right parenthesis:
\begin{verbatim}
\begin{enumerate}[label=\emph{\alph*}), ref=\emph{\alph*}]
\end{verbatim}

\begin{note}
  In both |label| and |ref|, the counters can be used as usual. So, and
  provided the current level is the second one:
\begin{verbatim}
\begin{enumerate}[label=\theenumi.\arabic*.]
\end{verbatim}
  or
\begin{verbatim}
\begin{enumerate}[label=\arabic{enumi}.\arabic*.]
\end{verbatim}  
\end{note}

\begin{note}
  The |label|s are \textit{not} accumulated to form the reference.
  If you want, say, something like 1.\textit{a} from 1) as first level
  and \textit{a}) as second level, you must set it with |ref|. You may
  use |\ref{level1}.\ref{level2}| with appropriate |ref| settings, but as
  Robin Fairbairns points out in the \TeX{} FAQ:
  \begin{quote}
  \dots{} [that] would be both tedious and error-prone. What is more, it 
  would be undesirable, since you would be constructing a visual 
  representation which is inflexible (you could not change all the 
  references to elements of a list at one fell swoop).
  \end{quote}
  This is sensible and I recommend to follow the advice, but sometimes
  you might want something like:
\begin{verbatim}
... subitem \ref{level2} of item \ref{level1} ...
\end{verbatim}
\end{note}

\begin{warning}
  The value of |ref| is a moving argument, and fragile commands must be
  protected \textit{except} the counters.
\end{warning}

\begin{desc}
|font=<commands>|\qquad|format=<commands>|
\end{desc}

Sets the label font. Useful when the label is changed with the optional
argument of |\item| and in \texttt{description}. The last command in
|<commands>| can take an argument with the item label. In
\texttt{description} class setting are in force, so you may want to begin
with |\normalfont|. A synonym is \texttt{format}. Actually, this key
may be used for any stuff to be executed at each |\item|, just before the
label.

\begin{desc}
|align=left|\qquad |align=right|\qquad |align=parleft|
\end{desc}

How the label is aligned (with relation to the label box edges).
Three values are possible: |left|, the default |right| and
|parleft| (a parbox of width |\labelwidth| with flush left
text).  The parameters controlling the label spacing should be
properly set, either by hand or more conveniently with the |*|
settings (see below):
\begin{verbatim}
\begin{enumerate}[label=\Roman*., align=left, leftmargin=*]
\end{verbatim}
When the label box is supposed to have its natural width, use
|left|.

\begin{desc}
|\SetLabelAlign{<value>}{<commands>}|
\end{desc}

New align types can be defined (or the existing ones redefined) with
|\SetLabelAlign|; the predefined values are equivalent
to:
\begin{verbatim}
\SetLabelAlign{right}{\hss\llap{#1}}
\SetLabelAlign{left}{#1\hfil}
\SetLabelAlign{parleft}{\strut\smash{\parbox[t]\labelwidth{\raggedright##1}}}
\end{verbatim}

\begin{example}
  Although primarily intended for the alignment, this commands has
  other uses (an example is the provided |parleft|). For example, with
  the following all labels with |align=right| are set as superscripts:
\begin{verbatim}
\SetLabelAlign{right}{\hss\llap{\textsuperscript{#1}}}
\end{verbatim}
  A new name is also possible, of course.
\end{example}

\begin{note}
  If the last thing in the definition is a skip (typically |\hfil|), it
  is removed sometimes by |description|. If for some reason you want to
  avoid this, just add |\null| at the end.
\end{note}

\begin{note}
  If you want the internal settings for \texttt{align} and \texttt{font}
  be ignored, you can override the \textsf{enumitem} definition of
  |\makelabel| in \texttt{before}:
\begin{verbatim}
\begin{description}[before={\renewcommand\makelabel[1]{\ref{##1}}}]
\end{verbatim}
  Alternatively, define a macro and use |\let|.
\end{note}

\subsection{Horizontal spacing of labels}

The horizontal space in the left margin of the current level is
distributed in the following way:\footnote{Admittedly, these figures
are not exactly the clearest possible, and I intend to improve them in
a future release.}
\begin{center}
\begin{tabular}{cc}
\fbox{\fbox{\strut \texttt{labelindent}}
  \fbox{\strut \texttt{labelwidth}}
  \fbox{\strut \texttt{labelsep} $-$ \texttt{itemindent}}}
&
\fbox{\strut\texttt{itemindent}}\\
\texttt{leftmargin}
\end{tabular}
\end{center}
Here |labelindent| is a new parameter introduced by \textsf{enumitem}, 
described below. The rest are those in standard \LaTeX.

Actually, the layout is more complex because the label box (ie,
|labelwidth|) could stick into the margin, which means |labelindent|
takes a negative value.

\begin{note}
  Since |\parindent| is not used as such inside lists, but instead is
  set internally to either |\itemindent| or |\listparindent|, when used
  as the value of a parameter \textsf{enumitem} returns the global
  value, i. e., the value it has outside the outermost list.
\end{note}

\begin{note}
\New{3.6} If you find these parameters baffling, you are not alone. You
can visualize them by writing |\DrawEnumitemLabel| just before the
first item (or in |first|), which draws 4 rules from top to bottom,
|labelindent|, |labelwidth|, |labelsep|, |itemindent| (thin if
positive, thick if negative); the |leftmargin| is marked with two
vertical rules.
\end{note}

\begin{desc}
|labelindent=<length>|\\
|\labelindent|
\end{desc}

This parameter is added in \textsf{enumitem} for the blank space from
the margin of the enclosing list/text to the left edge of the label
box. This means there is a redundancy because one of the parameters
depends on the others, i.e., it has to be computed from the other
values, as described below. By default, the computed value is
|labelindent|, even if explicitly set with some value (it defaults to
0~pt). So, if you are setting it to some value, very likely you want to
set some other parameter to |!| or |*|, because otherwise it is 
ignored.

There is a new counter length |\labelindent|.

The five parameters are related in the following way:
\[
|\leftmargin|+|\itemindent| = 
|\labelindent|+|\labelwidth|+|\labelsep|
\]

\begin{desc}
|left=<labelindent>|\\
|left=<labelindent> .. <leftmargin>|
\end{desc}

\New{3.6} This is a convenience key to set quickly the most common
layouts for the label. You may regard it as a sort or ``rule'' with two
tab stops: the start of the label and the start of the text (both with
relation to the normal side margin). With only |<labelindent>|, the
left margin (the ``start of text'') is computed with the |labelsep|. It
internally resorts to |widest|, so the restrictions of the letter with
relation to |description| also applies here: you might need change the
computed parameter (eg, |itemindent=*| with |align=left|).

\begin{example}
  Typical settings would be:
\begin{verbatim}
\begin{enumerate}[left= 0pt]
\begin{enumerate}[left= 0pt .. \parindent]
\begin{enumerate}[left= \parindent]
\begin{enumerate}[left= \parindent .. 2\parindent]
\begin{enumerate}[left= -\parindent .. 0pt]
\end{verbatim}
\end{example}

\begin{note}
  The label width is set to the default widest one. If there are lists
  with Arabic numerals $\ge 10$, you may want to set |widest|, too.
\end{note}

\begin{note}
|left=<labelindent>| sets |leftmargin=*|;
|left=<labelindent> .. <leftmargin>| sets |labelsep=*|.
\end{note}

\begin{desc}
|leftmargin=!|\qquad|itemindent=!|\qquad|labelsep=!|
\qquad|labelwidth=!|\qquad|labelindent=!|
\end{desc}

Sets which value is to be computed from the others. The default is
|labelindent=!|, but note some keys set another value (|wide| and
description |style|s). Computations are done after \textit{all} keys
have been read. Explicit values are not lost, and so with the following
hierarchical settings:
\begin{verbatim}
leftmargin=2em
labelindent=1em,leftmargin=!
labelindent=!
\end{verbatim}
|leftmargin| is again 2em and |labelindent| is the computed parameter.

\begin{note}
  With |align=right| (the default), |labelindent=!| and |labelwidth=!|
  behave similarly in practice. 
\end{note}

\begin{desc}
|leftmargin=*|\qquad|itemindent=*|\qquad|labelsep=*|
\qquad|labelwidth=*|\qquad|labelindent=*|
\end{desc}

Like before, but in addition |labelwidth| is first set to the width of
the current label, using the default value of \textit{0} in |\arabic*|,
\textit{viii} in |\roman*|, \textit{m} in |\alph*| and similarly in
uppercase forms (these values can be changed with |widest|, see below).
Examples are:
\begin{verbatim}
\begin{itemize}[label=\textbullet, leftmargin=*]
\begin{enumerate}[label=\roman*), leftmargin=*, widest=iii]
\begin{itemize}[label      = \textbullet,
                leftmargin = 2pc,
                labelsep   = *]
\begin{enumerate}[label       = \arabic*.,
                  labelindent = \parindent,
                  leftmargin  = 2\parindent, 
                  labelsep    = *]
\end{verbatim}

The most useful are |labelsep=*| and |leftmargin=*|. With the former
the item body begins at a fixed place (namely, |leftmargin|), while
with the latter begins at a variable place depending on the label (but
always the same within a list, of course).

\begin{note}
  Unfortunately, \LaTeX{} does not define a default |labelsep| to
  be applied to all lists---simply the current value is used.  With
  \textsf{enumitem} you can set default values for every list, as
  described below, and so, if you want to make sure |labelsep| is
  under your control, all you need is something like:
\begin{verbatim}
\setlist{labelsep=.5em}
\end{verbatim}
\end{note}

\begin{note}
  |labelwidth=*| and |labelwidth=!| are synonymous. Use them with care,
  because they may take negative values, which does not make sense (a
  warning is shown).
\end{note}

\begin{desc}
|widest=<string>|\qquad|widest*=<integer>|\qquad|widest|
\end{desc}

To be used in conjunction with the \texttt{*}-values, if desired. It
overrides the default value for the widest printed counter. Sometimes,
if lists are not very long, a value of |a| for |\alph| is more sensible
than the default |m|:
\begin{verbatim}
\begin{enumerate}[leftmargin=*, widest=a] % Assume standard 2nd level
\end{verbatim}
With no value, the default is restored. With |widest*|, the string is 
built using |<integer>| as the value of the counter (e.g., with 
|\roman|, 
|widest=viii| and |widest*=8| are the same).

Since |\value| does not return a string but a number, |widest| and the
|*| values cannot be used with it. However, with |widest*|,
being a number, it is allowed.

\New{3.6} It can be used with |itemize| and |description|, too. 
However, since the latter does some tricky formatting inside the label 
you might need change the computed parameter (eg, |itemindent=*| with 
|align=left|).

\begin{desc}
|labelsep*=<length>|
\end{desc}

Remember |labelsep| spans part of |leftmargin| and |itemindent| if the
latter is not zero. This is often somewhat confusing, so a new key is
provided---with \texttt{labelsep*} the value is reckoned from the left
margin (it just sets |\labelsep| and then adds |\itemindent| to it, but
in addition later changes to |itemindent| are taken into account):
\begin{center}
\begin{tabular}{cc}
\fbox{\fbox{\strut \texttt{labelindent}}
  \fbox{\strut \texttt{labelwidth}}
  \fbox{\strut \texttt{labelsep*}}}
&
\fbox{\strut\texttt{itemindent}}\\
\texttt{leftmargin}
\end{tabular}
\end{center}

\begin{desc}
|labelindent*=<length>|
\end{desc}

Like |labelindent|, but it is reckoned from the left margin in 
the current list and not from that in the enclosing list/text.

\begin{example}
  A first pattern aligns the label with the surrounding |\parindent|
  while the item body is indented depending on the label and a fixed
  |labelsep|:
\begin{verbatim}
labelindent = \parindent,
leftmargin  = *
\end{verbatim}
  A fairly frequent variant is aligning the label with the surrounding
  text (remember |labelindent| is |0pt| by default if it is not the
  computed parameter):
\begin{verbatim}
leftmargin = *
\end{verbatim}
  The former looks better in the first level while the latter seems
  preferable in subsequent ones. That can be easily set with
\begin{verbatim}
\setlist{leftmargin=*}
\setlist[1]{labelindent=\parindent} % Only the level 1
\end{verbatim}
\end{example}

\begin{example}
  A second pattern aligns the item body with the surrounding
  |\parindent|. In this case (remember |labelindent| is the computed
  parameter if not set):
\begin{verbatim}
leftmargin = \parindent
\end{verbatim}
\end{example}

\begin{example}
  A third pattern would be the label aligned with |\parindent|,
  and the item body with |2\parindent|:
\begin{verbatim}
labelindent = \parindent, 
leftmargin  = 2\parindent, 
itemsep     = *
\end{verbatim}
  Again, a variant would be the label aligned with the surrounding
  text, and the item body with |\parindent|:
\begin{verbatim}
leftmargin = \parindent,
itemsep    = *
\end{verbatim}
\end{example}

\subsection{Numbering, stopping, and resuming}

\begin{desc}
|start=<integer>|
\end{desc}
Sets the number of the first item.

\begin{desc}
|resume|
\end{desc}

The counter continues from the previous |enumerate|,
instead of being reset to 1.
\begin{verbatim}
\begin{enumerate}
\item First item.
\item Second item.
\end{enumerate}
Text.
\begin{enumerate}[resume]
\item Third item 
\end{enumerate}
\end{verbatim}

This is done locally. If you want global resuming, see next section on
series.

\begin{desc}
|resume*|
\end{desc}

Like |resume| but the options from the previous list are used, too.
This option must be restricted to the optional argument in a
environment (this is the only place where it makes sense). It should be
used sparingly---if you are using it often, then very likely you want
to define a new list (see \ref{s.clone}). Further keys are allowed, and
in this case the saved options are overridden by those in the current
list (i.e., the position of \texttt{resume*} does not matters). If
there is a series of a certain list with \texttt{resume*}, options are
taken from the list previous to the first one, except for
\texttt{start}.

\begin{example}
  For example:
\begin{verbatim}
\begin{enumerate}[resume*, start=1] % or [start=1, resume*]
\end{verbatim}
  uses the keys in the previous \texttt{enumerate}, but restarts the
  counter.  
\end{example}

\subsection{Series}

\begin{desc}
|series=<series-name>|\\
|<series-name>|\qquad|resume*=<series-name>|
\qquad|resume=<series-name>|
\end{desc}

Another method to continue lists is by means of the key
\texttt{series}, so that they behave like a unit. A list with key
\texttt{series} is considered the starting list and its settings are
stored \textit{globally}, so that they can be used later with
\texttt{resume}/\texttt{resume*}. All these keys take a value with the
series name (which must be different from existing keys):
\begin{itemize}
\item |resume=<series-name>| just continue numbering items in the
series,
\item |resume*=<series>| also applies the settings of the 
starting list,
\item |<series>|, i.e., the series name used as a key, is an 
alternative to |resume*=<series>|.
\end{itemize}

\begin{example}
  Consider:
\begin{verbatim}
\begin{enumerate}[label=\arabic*(a), leftmargin=1cm, series=l_after]
\item A
\item B
\end{enumerate}
\end{verbatim}
  You get: 1(a)  2(a). You can continue with:
\begin{verbatim}
\begin{enumerate}[label=\arabic*(b), resume*=l_after]
                 % or [label=\arabic*(b), l_after]
\item A
\item B
\end{enumerate}
\end{verbatim}
  You get: 3(b)  4(b). (But you can use |start=1|, if you like.)
\end{example}

Note you can add further arguments, which are executed after those
saved at the starting list and therefore take precedence over them --
in particular, |resume*| itself takes precedence over a |start| (e.g.,
|start=1|) in the the starting list.

\begin{note}
  Every time a series is started, several commands are defined 
  internally. Thus, to avoid wasting resources use the same name for 
  non-overlapping series.
\end{note}

\begin{warning}
  The package may introduce new keys in the future, so using directly
  |<series>| as a key is a potential source of forward
  incompatibilities. However, it's safe using a non-letter character
  other than hyphen or star in the key name (e.g., |!notes| or
  |m_steps|), as well as uppercase letters and digits, because
  \textsf{enumitem} will never use them. \New{3.7} If you have defined
  some series with an all lowercase name and a new conflicting key has
  been introduced, an alternative to changing their names is the
  package option |series=override| (the error message is
  \texttt{Invalid series name `key'}). With it series names take
  precedence over predefined keys -- but use it only when absolutely
  necessary.
\end{warning}

\subsection{Penalties}

\begin{desc}
|beginpenalty=<integer>|\qquad
|midpenalty=<integer>|\qquad |endpenalty=<integer>|
\end{desc}

Set the penalty at the beginning of a list, between items and at the
end of the list, respectively.  Please, refer to your \LaTeX{} or
\TeX{} manual about how penalties control page breaks.  Unlike other
parameters, when a list starts their values are not reset to the
default, thus they apply to the child lists.

\subsection{Injecting code}

\begin{desc}
|before=<code>| \qquad |before*=<code>|
\end{desc}

Execute code before the list starts (more precisely, in the second
argument of the |list| environment used to define them).  The
unstarred form sets the code to be executed, overriding any previous
value, while the starred one adds the code to the existing one (in
the setting hierarchy, see below, \textit{not} with relation to the
enclosing list/text).  It can contain, say, rules and text, but this
has not been extensively tested.  All calculations have been finished,
and you can access and manipulate the list parameters.

\begin{example}
  To have both margins (left and right) set to the widest label:
\begin{verbatim}
\setlist{leftmargin=*, before=\setlength{\rightmargin}{\leftmargin}}
\end{verbatim}
\end{example}

\begin{desc}
|after=<code>|\qquad|after*=<code>|
\end{desc}

Same, but just before the list ends.

\begin{desc}
|first=<code>|\qquad|first*=<code>|
\end{desc}

\New{3.6} Same, but as the very first thing in the list body, so that
\begin{verbatim}
\begin{itemize}[first=<code>]
\end{verbatim}
is the same as
\begin{verbatim}
\begin{itemize}
<code>
\end{verbatim}

\begin{example}
  With |first| you can define your own environments for displayed
  material. A trivial example is:
\begin{verbatim}
\newlist{letter}{itemize}{1}
\setlist[letter]{first=\item[]\itshape, rightmargin=\leftmargin}
\end{verbatim}
Here there is no need for a |label|, because it is not used.
\end{example}

\begin{desc}
|\EnumitemId|
\end{desc}

\New{3.7} To help in some tasks, a unique numeric identifier is
assigned to each list, returned by |\EnumitemId|.

\begin{example}
  Here is an example of how to combine a |\label| with |\EnumitemId|,
  and |after| to automatically set the width of the list label to the
  widest one (provided the ref is the same as the label):\footnote{See
  \texttt{https://tex.stackexchange.com/questions/29322/%
  how-to-make-enumerate-items-align-at-left-margin}.}
\begin{verbatim}
\SetEnumitemKey{widestlabel}
  {labelwidth = \widthof{\ref{enum-\EnumitemId}},
   after      = \label{enum-\EnumitemId}}
\end{verbatim}
Then just use the key |widestlabel|.
\end{example}

\begin{example}
  Reverse counting is also doable, but somewhat trickier, and we need
  some ``external'' help. Here is a possible solution, but not the
  only one (and very likely not even the best -- for example, |start|
  is in fact no-op).

\begin{verbatim}
\usepackage{calc,cleveref,crossreftools}
\crtrefundefinedtext{0}

\newcounter{revcount}
\newcommand\revcounter[1]{%
  \setcounter{revcount}{1+\crtcrefnumber{enum-\EnumitemId}-\value{#1}}}
\AddEnumerateCounter\revcounter\revcounter{} % the 2nd is dummy

\SetEnumitemKey{revarabic}
  {label = \revcounter*(\arabic{revcount}),
   ref   = (\arabic{revcount}),
   after = \label{enum-\EnumitemId}}
\end{verbatim}
  Note |ref| must be set separately, because |\revcounter|
  is not expandable.
\end{example}

\subsection{Description styles}

A key available in |description|.
\begin{desc}
|style=<name>|
\end{desc}

Sets the description \textit{style}. |<name>| can be any of the 
following:
\begin{description}
\item[|standard|] Like |description| in standard classes, although
with other classes it could be somewhat different.  The label is
boxed.  Sets |itemindent=!|.

\item[|unboxed|] Much like the standard |description|, but 
the label is not boxed to avoid uneven spacing and unbroken labels if 
they are long. Sets |itemindent=!|.

\item [|nextline|] If the label does not fit in the margin, the text
continues in the next line, otherwise it is placed in a box of width
|\leftmargin| $-$ |\labelsep|, i.e., the item body never sticks into
the left margin.  Sets |labelwidth=!|.

\item[|sameline|] Like |nextline| but if the label does not 
fit in the margin the text continues in the same line. Same as
|style=unboxed,labelwidth=!|.

\item[|multiline|] The label is placed in a parbox whose width is 
|leftmargin|, with several lines if necessary. Same as
|style=standard,align=parleft,labelwidth=!|. If you modify it, bear in
mind |align| cannot be set freely (for an internal optimization; in
particular, a horizontal box doesn’t make sense and can raise an
error).

\end{description}

\begin{warning}
\begin{enumerate}
\item Mixing boxed and unboxed labels has not a well-defined behavior.
\item When nesting list all combinations are allowed but not all make
  sense.
\item Nesting |nextline| lists is not supported (it works, but its
  behavior might change in the future, because the current one is not
  what one could expect).
\end{enumerate}
\end{warning}

\subsection{Compact lists}

\begin{desc}
|noitemsep|\qquad|nosep|
\end{desc}

The key |noitemsep| kills the space between items and paragraphs
(i.e., |itemsep=0pt| and |parsep=0pt|), while
|nosep| kills all vertical spacing.\footnote{The key 
\texttt{nolistsep}, now deprecated, introduced a thin stretch, which 
was not the intended behavior.}

\subsection{``Wide'' lists}

\begin{desc}
|wide|\\
|wide=<parindent>|
\end{desc}

With this convenience key, the leftmargin is null and the label is
part of the text---in other word, the items look like ordinary
paragraphs.\footnote{\texttt{fullwidth} is deprecated.} Here |labelsep|
sets the separation between the label and the first word.  It is
equivalent to
\begin{verbatim}
align=left, leftmargin=0pt, labelindent=\parindent,
listparindent=\parindent, labelwidth=0pt, itemindent=!
\end{verbatim}
With |wide=<parindent>| you may set at once another value instead of
|\parindent|.  Of course, these keys can be overridden after
|wide|, too; for example, remembering that with left-aligned labels
the text is pushed if the they are wider than |labelwidth|, you
can set |labelwidth=1.5em| for a minimal width, or instead of
|itemindent=!| you may prefer |itemindent=*|, which sets the
minimal width to that of widest label.  In level 2 you may prefer
|labelindent=2\parindent|, and so on.  You may also want to
combine it with |noitemsep| or |nosep|.

\subsection{\textsf{enumerate}-like labels}
\label{s.short}

\begin{desc}
|shortlabels| (package option)
\end{desc}

With the package option \texttt{shortlabels} you can use an
\textsf{enumerate}-like syntax, where |A|, |a|, |I|,
|i| and |1| stand for |\Alph*|, |\alph*|,
|\Roman*|, |\roman*| and |\arabic*|.  This is intended
mainly as a sort of compatibility mode with the \textsf{enumerate}
package, and therefore the following special rule applies: if the very
first option (at any level) is not recognized as a valid key, then it
will be considered a label with the \textsf{enumerate}-like syntax.  For
example:
\begin{verbatim}
\begin{enumerate}[i), labelindent=\parindent, labelsep=*]
...
\end{enumerate}
\end{verbatim}
You may want to set |ref|, too, if different from the label.

Although perhaps not so useful, you can omit |label=| in the
itemize environment under similar conditions, too:
\begin{verbatim}
\begin{itemize}[\textbullet]
...
\end{itemize}
\end{verbatim}

\begin{desc}
|\SetEnumerateShortLabel{<key>}{<replacement>}|
\end{desc}

With this command, you can define new keys (or redefine them), which is
particularly useful for enumerate to be adapted to specific
typographical rules or to extend it for non-Latin scripts. Here
|<replacement>| contains one of the starred versions of 
counters.

\begin{example}
  For example:
\begin{verbatim}
\SetEnumerateShortLabel{i}{\textsc{\roman*}}
\end{verbatim}
  redefines |i| so that items using this key are numbered with
  small caps roman numerals.
\end{example}

\begin{note}
  The key has to be a single character.
\end{note}

\subsection{Generic keys and values}

\begin{desc}
|\SetEnumitemKey{<key>}{<replacement>}|
\end{desc}

With this command you can create your own (valueless) keys. Keys so
defined can then be used like the others.

\begin{example}
  With
\begin{verbatim}
\SetEnumitemKey{midsep}{topsep=3pt, partopsep=0pt}
\end{verbatim}
  you may write
\begin{verbatim}
\begin{enumerate}[midsep]
\end{verbatim}
\end{example}

\begin{example}
  Another example is multicolumn lists, with \textsf{multicol}:
\begin{verbatim}
\SetEnumitemKey{twocol}{
  itemsep = 1\itemsep,
  parsep  = 1\parsep,
  before  = \raggedcolumns\begin{multicols}{2},
  after   = \end{multicols}}
\end{verbatim}
  Here, the settings for \texttt{itemsep} and \texttt{parsep} kill the
  stretch and shrink parts, which in this case improves the result. Of
  course, you may want to define a new list.
\end{example}

\begin{warning}
  The package may introduce new keys in the future, so
  |\SetEnumitemKey| is a potential source of forward incompatibilities.
  However, it's safe using a non-letter character other than hyphen or
  star in the key name (e.g., |:name| or |2_col|), as well as uppercase
  letters and digits, because \textsf{enumitem} will never use them.
\end{warning}

\begin{desc}
|\SetEnumitemValue{<key>}{<string-value>}{<replacement>}|
\end{desc}

This commands provides a further abstraction layer for the
|<key>=<value>| pairs.  With it you can define logical names which
are translated to the actual value.  For example, with:
\begin{verbatim}
\SetEnumitemValue{label}{numeric}{\arabic*.}
\SetEnumitemValue{leftmargin}{standard}{\parindent}
\end{verbatim}
you might say:
\begin{verbatim}
\begin{enumerate}[label=numeric, leftmargin=standard]
\end{verbatim}
So, you can left to the final design what |label=numeric| means.

\section{Inline lists}
\label{s.inline}

Inline lists are ``horizontal'' lists set as ordinary text inside a
paragraph. With this package you can create inline lists, as explained
below, with |\newlist|, which have their own labels and counters.
However, very often inline versions of standard lists, with the same
labeling schema, will be enough -- the package option |inline| does
that.

\begin{warning}
  Items are boxed, so floats are lost and nested lists are not allowed
  (remember many displayed elements are defined as lists). Display math
  is forbidden too, and due to an optimization done by \TeX{} when
  building lists, explicit hyphenation may be wrong.\footnote{A
  Knuthian ``premature optimization''? Who knows, but anyway Lua\TeX{}
  has removed it, so hyphenation with this engine should be correct.}
  There was a reason for this default setting, namely, this feature was
  mainly devised for short items (a few words), and the parameter
  |itemjoin*| could be useful for logical markup. To overcome these
  limitations, you may set |mode=unboxed|, described below.
\end{warning}

\begin{desc}
|inline| \qquad(package option)\\
\texttt{enumerate*}\qquad\texttt{itemize*}\qquad
\texttt{description*} \qquad(environments)
\end{desc}

With the package option \texttt{inline}, three environments for inline
lists are defined: \texttt{enumerate*}, \texttt{itemize*}, and
\texttt{description*}. They emulate the behavior of \textsf{paralist}
and \textsf{shortlst} in that labels and settings are shared with the
displayed (ie, ``normal'') lists \texttt{enumerate}, \texttt{itemize}
and \texttt{description}, respectively (however, remember resuming is
based on environment names, not on list types). This applies only to
those created with \texttt{inline} -- inline lists created with
|\newlist| as described below are independent and use their own labels
and settings.

\begin{note}
  Note |inline| is not required if you do not need the inline versions
  of standard lists, but instead you define your own standalone inline
  lists with |\newlist|.
\end{note}

\begin{warning}
  Settings for these three environments as defined by |inline| are
  shared with the displayed variants, so they cannot be redefined
  directly with |\newlist|. Trying to do it raises a cryptic error. If
  you need separate setting, define them with |\newlist| and not with
  |inline|.
\end{warning}

\begin{desc}
|itemjoin=<string>|\qquad|itemjoin*=<string>|
\qquad|afterlabel=<string>|
\end{desc}

Format is set with keys \texttt{itemjoin} (default is a space), and
\texttt{afterlabel} (default is |\nobreakspace|, ie, |~|).
An additional key is \texttt{itemjoin*}, which, if set, is used
instead of \texttt{itemjoin} before the last item. 

|itemjoin| is ignored in vertical mode (i.e., in mode unboxed
and just after a quote, a displayed list and the like).

\begin{example}
  With
\begin{verbatim}
before=\unskip{: }, itemjoin={{; }}, itemjoin*={{, and }}
\end{verbatim}
  the following punctuation between items is used:
  \begin{quote}
  Blah blah: (a) one; (b) two; (c) three, and (d) four. Blah blah
  \end{quote}
\end{example}

\begin{desc}
|mode=unboxed|\qquad|mode=boxed|
\end{desc}

If using floats, lists or displayed math inside inline lists is
important, use an alternative ``mode'', which you can activate with
\texttt{mode=unboxed} (the default is \texttt{mode=boxed}). With it,
floats may be used freely, but misplaced |\item|s are not caught and
\texttt{itemjoin*} is ignored (a warning is written to the log about
this fact).

\section{Global settings}

Global changes, to be applied to all of these list, are also
possible:
\begin{desc}
|\setlist[enumerate,<levels>]{<format>}|\\
|\setlist[itemize,<levels>]{<format>}|\\
|\setlist[description,<levels>]{<format>}|\\
|\setlist[<levels>]{<format>}|
\end{desc}
Where |<level>| is the list level (one or more) in |list|, and the
corresponding levels in |enumerate| and
|itemize|.\footnote{|\string\setenumerate|,
|\string\setitemize| and |\string\setdescription| are
deprecated.} With no |<levels>|, the format applies to all of them.
Here `list' does not mean any list but only the three ones handled by
this package, and those redefined by this package or defined with
|\newlist| (see below). For example:
\begin{verbatim}
\setlist{noitemsep}
\setlist[1]{labelindent=\parindent} % << Usually a good idea
\setlist[itemize]{leftmargin=*}
\setlist[itemize,1]{label=$\triangleleft$}
\setlist[enumerate]{labelsep=*, leftmargin=1.5pc}
\setlist[enumerate,1]{label = \arabic*.,
                      ref   = \arabic*}
\setlist[enumerate,2]{label = \emph{\alph*}),
                      ref   = \theenumi.\emph{\alph*}}
\setlist[enumerate,3]{label = \roman*),
                      ref   = \theenumii.\roman*}
\setlist[description]{font=\sffamily\bfseries}
\end{verbatim}
These setting are read in the following order: list, list at the
current level, enumerate/itemize/description, and
enumerate/itemize/description at the current level; if a key appears
several times with different values, the last one, i.e.,  the most
specific one, is applied.  If we are resuming a series or a list with
\texttt{resume*}, the saved keys are then applied.  Finally, the
optional argument (except \texttt{resume*}), if any, is applied.

\LaTeX{} provides a set of macros to change many of these parameters,
but setting them with the package is more consistent and sometimes
more flexible at the cost of being more ``explicit'' (and verbose).

The list specification can contain variables and counters, provided
they are expandable, and counters are \textsf{calc}-savvy, so that if
you load this package you can write things like:
\begin{verbatim}
\newcount{toplist}
\setcount{toplist}{1}
\newcommand{\mylistname}{enumerate}
\setlist[\mylistname,\value{toplist}+1]{labelsep=\itemindent+2em]
\end{verbatim}
This allows defining lists with the help of loops.

\begin{warning}
  It seems there is no way to catch a misspelled name in |\setlist| or
  |\newlist|, and a meaningless error ``Missing number, treated as
  zero'' is raised.
\end{warning}

\section{Size dependent settings}
\label{s:sized}

\New{3.6} For settings depending on the font size, in most cases all
you need are relative units like |ex| or |em|. Sometimes, you may want
discrete steps, and a special syntax allows them.

The following package option is required for making use of this
feature.

\begin{desc}
|sizes| (package option)
\end{desc}

Lengths can contain size-based settings as follows (the value before
the first \texttt{<} is a default).
\begingroup
\makeatletter
\renewcommand\verbatim@font{\normalfont\ttfamily}
\begin{verbatim}
\setlist{
  topsep      = 20pt <-10> 6pt          <10-> 40pt,
  leftmargin  =      <-10> 0cm <10> 1cm <10-> 2cm ,
  rightmargin =      <-10> 0cm <10> 1cm <10-> 2cm ,
  }
\end{verbatim}
\endgroup

Names are accepted, too: |script|, |tiny|, |footnote|, |small|,
|normal|, |large|, |Large|, |LARGE|, |huge|, |Huge| (ie, remove `size'
from the \LaTeX{} name if necessary). For example:
\begingroup
\makeatletter
\renewcommand\verbatim@font{\normalfont\ttfamily}
\begin{verbatim}
\setlist{
  topsep      = 20pt <-normal> 6pt              <normal-> 40pt,
  leftmargin  =      <-normal> 0cm <normal> 1cm <normal-> 2cm ,
  rightmargin =      <-normal> 0cm <normal> 1cm <normal-> 2cm ,
  }
\end{verbatim}
\endgroup

Single values take precedence over ranges (i. e., specific takes
precedence over generic). In ranges, the last match wins. The range
|a-b| is $a \le \mbox{size} < b$ (the lower bound is included, but not
the upper one). These rules allow in the examples above the setting for
|10| or |normal| in the logical place. Remember nominal sizes are not
always the real sizes -- for example, with option |11pt|, |\normalsize|
(and |normal|) is 10.95. You may precede a value with several single
qualifiers like
\texttt{<}|small|\texttt{><}|normal|\texttt{>}|12pt|.\footnote{Note
this syntax follows closely that of \texttt{\string\DeclareFontShape},
except in the precedence of single values.} A value before the first
\texttt{<..>} is considered a default value.

\begin{note}
  For efficiency reasons, named sizes are assigned only once, when
  \textsf{enumitem} is loaded, in the assumption they are set by the
  class, or a local style loaded previously.
\end{note}

\begin{desc}
|\SetEnumitemSize{<name>}{<selector>}|
\end{desc}

\New{3.7} If sizes are modified after loading \textsf{enumitem} or you
are using a class with non standard sizes (or even you just want
another names), they can be set or reset with the following tool.

\begin{example}
A trivial example:
\begin{verbatim}
\SetEnumitemSize{normal}{\normalsize}
\SetEnumitemSize{large}{\large}
\end{verbatim}
\end{example}

\begin{desc}
|\setlist|\texttt{\string<}|<size>|\texttt{\string>}%
   |[<names>,<levels>]{<keys/values>}|
\end{desc}

\New{3.7} An extension to |\setlist| described below which adds the
definitions, but only for the given size (either single or a range).
The precedence rules for sizes also apply here (so that the order of
|\setlist|'s are relevant), and size dependent keys as defined by this
procedure take precedence over the rest of the keys. For example:
\begingroup
\makeatletter
\renewcommand\verbatim@font{\normalfont\ttfamily}
\begin{verbatim}
\setlist<-normal>[enumerate]{nosep}
\end{verbatim}
\endgroup

However, only a size qualifier is accepted in each |\setlist|.

\section{Cloning the basic lists}
\label{s.clone}

\begin{desc}
|\newlist{<name>}{<type>}{<max-depth>}|\\
|\renewlist{<name>}{<type>}{<max-depth>}|
\end{desc}

The three lists can be cloned so that you can define ``logical''
environments behaving like them. To define a new lists (or redefine a
existing one), use |\newlist| (or |\renewlist|), where |<type>| is
|enumerate|, |itemize| or |description|. Inline lists have types
\texttt{enumerate*}, \texttt{itemize*}, and \texttt{description*}.

\begin{note}
  The inline |<type>|s  are available always, even without the package
  option |inline|, which just defines three environments of the
  corresponding types with those names.
\end{note}

If |<type>| is |enumerate|, a set of counters with names |<name>i|,
|<name>ii|, |<name>iii|, |<name>iv|, etc.  (depending on |<max-depth>|)
is defined.

Then you can use those counters in labels; e. g., if you have defined a
list named \texttt{steps}, you can define a label with:
\begin{verbatim}
label=\arabic{stepsii}.\arabic{stepsi}
\end{verbatim}

\begin{warning}
  Don't use an arbitrarily large number for |<max-depth>|, to avoid
  creating too many counters and related macros.
\end{warning}

\begin{warning}
  For consistency with the counter naming schema in \LaTeX, list levels
  are also named internally with a roman numeral, ie, |<list>i|,
  |<list>ii|, |<list>iii|, etc. For this reason (both counter and list
  names), defining two lists as, say, |books| and |booksi| leads to
  unexpected results (currently without any warning, which should be
  fixed). 
  % TODO - Perhaps I must change the internal names, so that 
  % this restriction doesn't apply to itemize and description.]
\end{warning}

\begin{desc}
|\setlist[<names>,<levels>]{<keys/values>}|\\
|\setlist*[<names>,<levels>]{<keys/values>}|
\end{desc}

After creating a list, you can (in fact you
must, at least the label) set the new list with |\setlist|:
\begin{verbatim}
\newlist{ingredients}{itemize}{1}
\setlist[ingredients]{label=\textbullet}
\newlist{steps}{enumerate}{2}
\setlist[steps,1,2]{label=(\arabic*)}
\end{verbatim}
Names in the optional argument of |\setlist| say which lists applies the
settings to, and numbers say the level (it is |calc|-savvy).  Several
lists and/or several levels can be given, and all combinations are
set; e.g.:
\begin{verbatim}
\setlist[enumerate,itemize,2,3]{...}
\end{verbatim}
\noindent sets enumerate/2, enumerate/3, itemize/2 and itemize/3. 
No number (or 0) means ``all levels'' and no name means ``all lists''; no 
optional argument means ``all lists at all levels''.

The starred form |\setlist*| adds the settings to the previous ones.
You may restrict the additions to a certain font size, as explained in
section \ref{s:sized}. It must be noted the latter have a higher
precedence than the starred ones (i. e., settings added for some sizes
take precedence over settings added for all sizes, so that the most
specific value for a key is applied).

\begin{desc}
|\setlistdepth{<integer>}|
\end{desc}

By default, \LaTeX{} has a limit of 5 nesting levels, but when 
cloning lists this value may be too short, and therefore you may want 
to set a new value. In levels below the 5th (or the deepest defined by a 
class), the settings of the last are used (i.e., |\@listvi|).

\section{More about counters}

\subsection{New counter representation}

\begin{desc}
|\AddEnumerateCounter{<LaTeX command>}{<internal command>}{<widest label>}|
\end{desc}

``Registers'' a counter representation so that \textsf{enumitem}
recognizes it.  Intended mainly for non Latin scripts, but also useful
in Latin scripts. 

\begin{example}
The following example defines a new counter with named ordinals:
\begin{verbatim}
\makeatletter
\def\ctext#1{\expandafter\@ctext\csname c@#1\endcsname}
\def\@ctext#1{\ifcase#1\or First\or Second\or Third\or
Fourth\or Fifth\or Sixth\fi}
\makeatother
\AddEnumerateCounter{\ctext}{\@ctext}{Second}
\end{verbatim}
\end{example}

\begin{note}
  The counter names can contain |@| even if not a letter without
  raising an error, as shown in the example above.
\end{note}

A starred variant allows to give a number instead of a string as the 
widest label.

\begin{example}
  If the widest label is that corresponding to the value 2:
\begin{verbatim}
\AddEnumerateCounter*{\ctext}{\@ctmoreext}{2}
\end{verbatim}
\end{example}

This variant is to be preferred if the representation is not a plain 
string but it is styled, e.g., with small caps. 

\begin{example}
  An example for Russian is:
\begin{verbatim}
\AddEnumerateCounter*{\asbuk}{\c@asbuk}{7}
\end{verbatim}
\end{example}

\subsection{Restarting \texttt{enumerate}s}

\begin{desc}
|\restartlist{<list-name>}|
\end{desc}

Currently you can get a continuous numbering through a document with:
\begin{verbatim}
\setlist[enumerate]{resume}
\end{verbatim}
|\restartlist| has been added for restarting the counter in the middle
of the document. For example, you could emit a |\restartlist| when 
chapters start, so that there is a continuous numbering through every 
chapter.

\begin{warning}
It is based solely in the list \textit{name}, \textit{not} the list
\textit{type}, which means \texttt{enumerate*} as defined with the
package option \texttt{inline} is not the same as \texttt{enumerate},
because its name is different.
\end{warning}

\section{Package options}

Besides |inline|, |ignoredisplayed|, |sizes|, |series=override| and
|shortlabels|, the following option is available.

\begin{desc}
|loadonly|
\end{desc}

With this package option the package is loaded but the three
lists are not redefined. You can create your own lists, yet, or
even redefine the existing ones.

\section{The trivlist issue}

\LaTeX{} uses a simplified version of |list| named |trivlist| to set
displayed material, like |center|, |tabbing|, |theorem|, etc., even if
conceptually they are not lists. Unfortunately, |trivlist| uses the
current list settings, which has the odd side effect that changing the
vertical spacing of lists also changes sometimes the spacing in these
environments.

This package modifies |trivlist| so that the default settings for 
the current level (ie, those set by the corresponding |clo| 
files) are set again. In standard \LaTeX{} that is usually redundant, 
but if we want to fine tune lists, not resetting the default values 
could be a real issue (particularly if you use the |nosep| 
option).

A minimal control of vertical spacing has been made possible 
with\footnote{|\string\setdisplayed| is deprecated.}
\begin{itemize}
\item |\setlist[trivlist,<level>]{<keys/values>}|
\end{itemize}
but |trivlist| itself, which is not used directly very often, does not
accept an optional argument. This feature is not intended as a
full-fledge |trivlist| formatter.

If for some reason you do not want to change |trivlist| and preserve
the original definition, you can use the package option
|ignoredisplayed|.

\New{3.6} If, on the other hand, you want to also apply the changes for
all lists to trivlists, just set the package option |includedisplayed|.

\section{Samples}

\expandafter\ifx\csname setenumerate\endcsname\relax

Please, install first the package and then typeset this document again.

\else

In these samples we set |\setlist{noitemsep}|

\setlist{noitemsep}
\small

\newcommand{\newsample}{\vskip6pt\goodbreak\hrule height 1pt\vskip6pt}
\newcommand{\samplesep}{\vskip6pt\goodbreak\hrule\vskip6pt}
\newbox\vsep
\setbox\vsep\hbox{\vrule height 2ex depth 16ex width 1pt}
\dp\vsep0pt
\newcommand\showsep{\leavevmode\llap{\copy\vsep}}

\newsample

\begin{verbatim}
En un lugar de la Mancha, de cuyo nombre no quiero acordarme,
no ha mucho tiempo que viv\'{\i}a un hidalgo de los de
\begin{enumerate}[labelindent=\parindent,leftmargin=*]
  \item lanza en astillero,
  \item adarna antigua,
  \item roc\'{\i}n flaco, y
  \item galgo corredor.
\end{enumerate}
Una olla de algo m\'{a}s vaca que carnero, salpic\'{o}n las m\'{a}s
noches, duelos y quebrantos los s\'{a}bados...
\end{verbatim}

The rule shows |labelindent|. 

\samplesep

\showsep En un lugar de la Mancha, de cuyo nombre no quiero acordarme,
no ha mucho tiempo que viv\'{\i}a un hidalgo de los de
\begin{enumerate}[labelindent=\parindent,leftmargin=*]
\item lanza en astillero,
\item adarna antigua,
\item roc\'{\i}n flaco, y
\item galgo corredor.
\end{enumerate}
Una olla de algo m\'{a}s vaca que carnero, salpic\'{o}n las m\'{a}s
noches, duelos y quebrantos los s\'{a}bados...

\newsample

With |\begin{enumerate}[leftmargin=*] % labelindent=0pt by default|. 

The rule shows |labelindent|.

\samplesep

\noindent\showsep\hskip\parindent En un lugar de la Mancha, de cuyo nombre no quiero acordarme,
no ha mucho tiempo que viv\'{\i}a un hidalgo de los de
\begin{enumerate}[leftmargin=*]      
\item lanza en astillero,
\item adarna antigua,
\item roc\'{\i}n flaco, y
\item galgo corredor.
\end{enumerate}
Una olla de algo m\'{a}s vaca que carnero, salpic\'{o}n las m\'{a}s
noches, duelos y quebrantos los s\'{a}bados...

\newsample

With |\begin{enumerate}[leftmargin=\parindent]|.

The rule shows |leftmargin|.

\samplesep

\showsep En un lugar de la Mancha, de cuyo nombre no quiero acordarme,
no ha mucho tiempo que viv\'{\i}a un hidalgo de los de
\begin{enumerate}[leftmargin=\parindent]
\item lanza en astillero,
\item adarna antigua,
\item roc\'{\i}n flaco, y
\item galgo corredor.
\end{enumerate}
Una olla de algo m\'{a}s vaca que carnero, salpic\'{o}n las m\'{a}s
noches, duelos y quebrantos los s\'{a}bados...

\newsample

With |\begin{enumerate}[labelindent=\parindent,|\allowbreak
| leftmargin=*,|\allowbreak| label=\Roman*.,|\allowbreak
| widest=III,|\allowbreak| align=left]|.

The rule shows |labelindent|. Note 

\samplesep

\showsep En un lugar de la Mancha, de cuyo nombre no quiero acordarme,
no ha mucho tiempo que viv\'{\i}a un hidalgo de los de
\begin{enumerate}[labelindent=\parindent, leftmargin=*,
                  label=\Roman*., widest=III, align=left]
\item lanza en astillero,
\item adarna antigua,
\item roc\'{\i}n flaco, y
\item galgo corredor.
\end{enumerate}
Una olla de algo m\'{a}s vaca que carnero, salpic\'{o}n las m\'{a}s
noches, duelos y quebrantos los s\'{a}bados...

\newsample

With |\begin{enumerate}[label=\fbox{\arabic*}]|. A reference to
the first item is \ref{i:first}

\samplesep

En un lugar de la Mancha, de cuyo nombre no quiero acordarme,
no ha mucho tiempo que viv\'{\i}a un hidalgo de los de
\begin{enumerate}[label=\fbox{\arabic*}]
\item \label{i:first}lanza en astillero,
\item adarna antigua,
\item roc\'{\i}n flaco, y
\item galgo corredor.
\end{enumerate}
Una olla de algo m\'{a}s vaca que carnero, salpic\'{o}n las m\'{a}s
noches, duelos y quebrantos los s\'{a}bados...

\newsample

With nested lists.

\samplesep

\begin{verbatim}
En un lugar de la Mancha, de cuyo nombre no quiero acordarme,
no ha mucho tiempo que viv\'{\i}a un hidalgo de los de
\begin{enumerate}[label=(\alph*), labelindent=\parindent,
     leftmargin=*, start=12]
\item lanza en astillero,
\begin{enumerate}[label=(\alph{enumi}.\roman*), leftmargin=*, start=7]
\item adarna antigua,
\end{enumerate}
\item roc\'{\i}n flaco, y
\begin{enumerate}[label=(\alph{enumi}.\roman*), leftmargin=*, resume]
\item galgo corredor.
\end{enumerate}
\end{enumerate}
Una olla de algo m\'{a}s vaca que carnero, salpic\'{o}n las m\'{a}s
noches, duelos y quebrantos los s\'{a}bados...
\end{verbatim}

En un lugar de la Mancha, de cuyo nombre no quiero acordarme,
no ha mucho tiempo que viv\'{\i}a un hidalgo de los de
\begin{enumerate}[label=(\alph*), labelindent=\parindent,
     leftmargin=*, start=12]
\item lanza en astillero,
\begin{enumerate}[label=(\alph{enumi}.\roman*), leftmargin=*, start=7]
\item adarna antigua,
\end{enumerate}
\item roc\'{\i}n flaco, y
\begin{enumerate}[label=(\alph{enumi}.\roman*), leftmargin=*, resume]
\item galgo corredor.
\end{enumerate}
\end{enumerate}
Una olla de algo m\'{a}s vaca que carnero, salpic\'{o}n las m\'{a}s
noches, duelos y quebrantos los s\'{a}bados...

\newsample

\begin{verbatim}
En un lugar de la Mancha, de cuyo nombre no quiero acordarme,
no ha mucho tiempo que viv\'{\i}a un hidalgo de los de
\begin{description}[font=\sffamily\bfseries, leftmargin=3cm,
    style=nextline]
  \item[Lo primero que ten\'{\i}a el Quijote] lanza en astillero,
  \item[Lo segundo] adarna antigua,
  \item[Lo tercero] roc\'{\i}n flaco, y
  \item[Y por \'{u}ltimo, lo cuarto] galgo corredor.
\end{description}
Una olla de algo m\'{a}s vaca que carnero, salpic\'{o}n las m\'{a}s
noches, duelos y quebrantos los s\'{a}bados...
\end{verbatim}

\samplesep

En un lugar de la Mancha, de cuyo nombre no quiero acordarme,
no ha mucho tiempo que viv\'{\i}a un hidalgo de los de
\begin{description}[font=\sffamily\bfseries, leftmargin=3cm,
    style=nextline]
\item[Lo primero que ten\'{\i}a el Quijote] lanza en astillero,
\item[Lo segundo] adarna antigua,
\item[Lo tercero] roc\'{\i}n flaco, y
\item[Y por \'{u}ltimo, lo cuarto] galgo corredor.
\end{description}
Una olla de algo m\'{a}s vaca que carnero, salpic\'{o}n las m\'{a}s
noches, duelos y quebrantos los s\'{a}bados...

\newsample

Same, but with |sameline|.

\samplesep

En un lugar de la Mancha, de cuyo nombre no quiero acordarme,
no ha mucho tiempo que viv\'{\i}a un hidalgo de los de
\begin{description}[font=\sffamily\bfseries, leftmargin=3cm,
    style=sameline]
\item[Lo primero que ten\'{\i}a el Quijote] lanza en astillero,
\item[Lo segundo] adarna antigua,
\item[Lo tercero] roc\'{\i}n flaco, y
\item[Y por \'{u}ltimo, lo cuarto] galgo corredor.
\end{description}
Una olla de algo m\'{a}s vaca que carnero, salpic\'{o}n las m\'{a}s
noches, duelos y quebrantos los s\'{a}bados...

\newsample

Same, but with |multiline|. Note the text overlaps if the item body is
too short.

\samplesep

En un lugar de la Mancha, de cuyo nombre no quiero acordarme,
no ha mucho tiempo que viv\'{\i}a un hidalgo de los de
\begin{description}[font=\sffamily\bfseries, leftmargin=3cm,
    style=multiline]
\item[Lo primero que ten\'{\i}a el Quijote] lanza en astillero,
\item[Lo segundo] adarna antigua,
\item[Lo tercero] roc\'{\i}n flaco, y
\item[Y por \'{u}ltimo, lo cuarto] galgo corredor.
\end{description}
Una olla de algo m\'{a}s vaca que carnero, salpic\'{o}n las m\'{a}s
noches, duelos y quebrantos los s\'{a}bados...

\fi

\normalsize

\section{Afterword}

\subsection{\LaTeX{} lists}

As it is well known, \LaTeX{} predefines three lists:
\texttt{enumerate}, \texttt{itemize} and \texttt{description}.  This
is a very frequent classification which can also be found in, say,
HTML. However, there is a more general model based in three
fields---namely, label, title, and body---, so that enumerate and
itemize has label (numbered and unnumbered) but no title, while
description has title but no label.  In this model, one can have a
description with entries marked with labels, as for example (of 
course, this simple solution is far from satisfactory):
\begin{verbatim}
\newcommand\litem[1]{\item{\bfseries #1,\enspace}}
\begin{itemize}[label=\textbullet]
\litem{Lo primero que ten\'{\i}a el Quijote} lanza en astillero,
... etc.
\end{verbatim}

\vskip6pt
\goodbreak
\hrule
\vskip6pt

\newcommand\litem[1]{\item{\bfseries #1,\enspace}}
En un lugar de la Mancha, de cuyo nombre no quiero acordarme,
no ha mucho tiempo que viv\'{\i}a un hidalgo de los de
\begin{itemize}[label=\textbullet]
\litem{Lo primero que ten\'{\i}a el Quijote} lanza en astillero,
\litem{Lo segundo} adarna antigua,
\litem{Lo tercero} roc\'{\i}n flaco, y
\litem{Y por \'{u}ltimo, lo cuarto} galgo corredor.
\end{itemize}

\vskip6pt
\goodbreak
\hrule
\vskip6pt

% This format in not infrequent at all and a tool for defining them is
% on the way and at a very advanced stage. It has not been include in
% version 3.0 because I'm not sure if the proper place is this package
% or \textsf{titlesec} and it is not stable enough yet.

\subsection{Known issues}

\begin{itemize}

\item
List resuming is based on environment names, and when a
|\newenvironment| contains a list you may want to use |\begin{<list>}|
and |\end{<list>}|. Using the corresponding low-level commands (ie,
|\<list>| and |\end<list>|) is not an error, but it is your
responsibility to make sure the result is correct.

\item The behavior of mixed boxed labels (including enumerate and
itemize) and unboxed labels is not well-defined.  The same applies to
boxed and unboxed inline lists (which could even raise an error).

\item
Similarly, resuming a series and a list at the same time is allowed,
too, but again its behavior is not well-defined.

\item (3.5.2) An incompatibility with 2.x has popped up -- if you were
using the optional argument to pass a value to a |\ref| or other 
macro requiring expandable macros, an error is raised. A quick fix 
is letting |\makelabel| to |\descriptionlabel| in \texttt{before}.

\end{itemize}

\subsection{What's new in 3.0}

\begin{itemize}
\item Inline lists, with keys to set how items are joined (ie, the
punctuation between items).  Two modes are provided: |boxed|  and
|unboxed|.

\item |\setlist| is \textsf{calc}-savvy (eg, for use in loops),
and you can set different lists and levels at once.  \item All lengths
related to labels can take the value |*| (and not only
|labelsep| and |leftmargin|).  Its behavior has been made
consistent and there is new value |!| which does not compute the
widest label.

\item With |\restartlist{<list-name>}|, list counters can be restarted (in
case you are using |resume|).

\item |resume*| can be combined with other keys.

\item Lists can be gathered globally using series, so that they are
considered a single list. To start a series just use
|series=<series-name>| and then resume it with |resume=<series-name>|
or |resume*=<series-name>|.

\item The ``experimental'' |fullwidth| has been replaced by a new key
|wide|.

\item|\SetLabelAlign| defines new align values.

\item You can define ``abstract'' values (eg, |label=numeric|) and
new keys.
\end{itemize}

\begin{itemize}
\item (3.2) |start| and |widest*| are \textsf{calc}-savvy.
\item (3.2) |\value| can be used with |widest*|.
\item (3.2) Some internal restrictions in |\arabic| and the like
has been removed.  It is more flexible at the cost of having a more
``relaxed'' error checking.
\end{itemize}
\subsection{Bug fixes}

\begin{itemize}
\item Star values (eg, |leftmargin=*|) could not be overridden
and new values were ignored.
\item |nolistsep| as the first of several keys was not always
recognized and therefore treated like a short label
(i.e., |nol\roman*stsep|).
\item |labelwidth| did not always work (when there was a prior
|widest| and |*|)
\item With |align=right| the label and the following text could
overlap.
\item |description| did not get the correct list level.
\item At some point (2.x?) |\value*| stopped working.
\item (3.1) Unfortunately, \textsf{xkeyval} ``kills'' 
\textsf{keyval}, so the latest has been replicated in 
\textsf{enumitem}.
\item (3.3) Fixes a serious bug -- with |*| neither 
|itemize| nor |description| worked.
\item (3.4) Fixes bad spacing in mode boxed (misplaced |\unskip| 
before the first item and wrong spacefactor between items).
\item (3.4) |nolistsep| did not work as intended, but since the
error has been there for several years, a new key |nosep| is
provided.
\item (3.4) The issue with |nolistsep| with |shortlabels| 
(see above) was not fixed in all cases. Hopefully now it is.
\item (3.5.0) Fixed the fix related to the spacefactor between items.
\item (3.5.0) Fixed a problem with nested boxed inline lists.
\item (3.5.1) \texttt{resume*} only worked once, and subsequent ones
bahaved like \texttt{resume}.
\item (3.5.2) Fixed |\setlist*|, which didn't work.
\end{itemize}

\subsection{Acknowledgements}

I wish to thank particularly the comments and suggestions from Lars
Madsen, who has found some bugs, too.

\end{document}

MIT License
-----------

Permission is hereby granted, free of charge, to any person obtaining a
copy of this software and associated documentation files (the
"Software"), to deal in the Software without restriction, including
without limitation the rights to use, copy, modify, merge, publish,
distribute, sublicense, and/or sell copies of the Software, and to
permit persons to whom the Software is furnished to do so, subject to
the following conditions:

The above copyright notice and this permission notice shall be included
in all copies or substantial portions of the Software.

THE SOFTWARE IS PROVIDED "AS IS", WITHOUT WARRANTY OF ANY KIND, EXPRESS
OR IMPLIED, INCLUDING BUT NOT LIMITED TO THE WARRANTIES OF
MERCHANTABILITY, FITNESS FOR A PARTICULAR PURPOSE AND NONINFRINGEMENT.
IN NO EVENT SHALL THE AUTHORS OR COPYRIGHT HOLDERS BE LIABLE FOR ANY
CLAIM, DAMAGES OR OTHER LIABILITY, WHETHER IN AN ACTION OF CONTRACT,
TORT OR OTHERWISE, ARISING FROM, OUT OF OR IN CONNECTION WITH THE
SOFTWARE OR THE USE OR OTHER DEALINGS IN THE SOFTWARE.
